% -*- root: ../main.tex -*-
\chapter{Processo di Sviluppo}


\section{Metodologia di Sviluppo}
Prima di iniziare a lavorare è stato fondamentale accordarsi su come affrontare in gruppo un progetto mediamente complesso come questo. Si è pensato subito di utilizzare una metodologia agile, per evitare confusioni e lavorare efficientemente in team. Si è optato quindi per la metodologia Scrum.\vspace{0.5cm}

    Scrum è una metodologia agile, incrementale e iterativa per lo sviluppo di prodotti, applicazioni e servizi. E' una modalità strutturata e pianificata.\\
    Scrum si basa sull’empirismo, ovvero sul concetto che la conoscenza derivi dall’esperienza e che le decisioni vadano prese alla luce di ciò che si conosce. I tre pilastri che sostengono l’empirismo sono: trasparenza, ispezione e adattamento.\\
    Ovvero, tutti gli aspetti del lavoro devono essere visibili ai responsabili del risultato finale (trasparenza). Per rendere trasparenti questi elementi, il Team Scrum ispeziona di frequente il prodotto mentre lo sta sviluppando (ispezione). Così il processo e il prodotto possono essere adattati immediatamente nel caso di nuove esigenze o di condizioni mutate del mercato (adattamento).\\ 
    Perchè questo è il senso di Scrum, far lavorare tutto il team insieme, in modo coordinato e organizzato.\\
    Come tutte le metodologie Agile, si basa sulla divisione del progetto in più fasi, chiamate Sprint.\\
Ad ogni Sprint il gruppo di lavoro presenta nuove funzionalità, operative e implementabili. Si configura così un sistema iterativo che consente di incrementare poco alla volta, ma molto di frequente, le funzionalità del progetto, verificando costantemente allo stesso tempo l'andamento complessivo.\vspace{0.5cm}

Siamo sempre stati fedeli nell'utilizzare tale metodologia, ma durante il progetto spesso è stata cambiata la durata degli Sprint, per il semplice fatto che per vari motivi non si riusciva a portare a termine i compiti assegnati entro la settimana. Ognuno dei vari Sprint ha quindi avuto una durata di svolgimento variabile. Il risultato è stato sicuramente l'allungamento dei tempi nella realizzazione del progetto, ma non abbiamo mai smesso di seguire tale metodo di lavoro e di confrontarci.


\section{Gestione di Progetto}
In questa sezione viene dettagliatamente spiegato come il progetto è stato organizzato. Gli strumenti e tecnologie con cui è scelto di procedere verranno elencati assieme alla descrizione della metodologia.
    \paragraph{Organizzazione: }
    A inizio progetto sono stati stilati insieme i primi requisiti di business e funzionali, sono state delineate le prime user stories, ragionando sui problemi che avrebbe dovuto affrontare tale tipologia di lavoro.
   Si è deciso di suddividere il lavoro in Sprint, ognuno con durata teorica di una settimana. Ad ogni Sprint corrispondevano diverse attività, anche chiamate issue. Tale tempistica, come accennato sopra, poi non è sempre stata rispettata. Spesso si sono riscontrati problemi durante lo svolgimento di un task e i tempi si sono allungati.\\
    All'interno del team Daniele Tentoni è stato sicuramente colui che ha svolto il ruolo dello Scrum Master oltre a far parte del gruppo di sviluppo: egli ha guidato il team per facilitare lo sviluppo di software funzionante, assicurando anche la corretta applicazione dei processi scrum.\\
    Lo Sprint con le varie attività da svolgere veniva stilato solitamente durante gli incontri, e le decisioni su cosa fare venivano prese collegialmente. Il Product Backlog è stato quindi fatto insieme.\\ 
    Si è proceduto in questo modo: si è iniziato a produrre la parte Server e, una volta sviluppate le principali funzionalità di quest'ultimo con i relativi test, si è poi passati a curare la parte client e la parte grafica. Nel frattempo si è anche iniziata a collegare la parte backend con quella frontend. Per ultimo, sono stati fatti anche gli User Acceptance Test.\\
    Ognuno era delegato al testing e agli User Acceptance Test sulla propria parte di codice prodotta,i test sono interni al progetto. 
    
    
    
    \paragraph{Gantt Chart: } 
    A partire da inizio progetto si è stilato un diagramma di Gantt, uno strumento utile per la pianificazione dei progetti. Attraverso una panoramica dei compiti programmati, ci ha permesso di venire a conoscenza dei compiti e delle rispettive scadenze.
    
    \paragraph{Licensing: } Per quanto riguarda la gestione delle licenze, è stato scelta, tra le varie disponibili, la licenza GNU General Public Service. E' una licenza copyleft, ovvero un tipo di licenza che viene applicata ai software open source. Il programma può essere quindi distribuito liberamente, modificato e migliorato.
    
    \paragraph{Versioning: }
    Per il versioning ci si è avvalsi di Git e come piattaforma web di Github che incorpora le funzionalità di controllo di versione di git. 
    
    \paragraph{Github Issues and Project Boards: }
    Per organizzare il lavoro, compilare e assegnare i tasks sono state utilizzate le Github Issues dove è possibile pianificare e tenere traccia del proprio operato aggiungendo varie issue al progetto, ognuna corrispondente a un task. Ad ogni issue è possibile aggiungere vari elementi, labels e il nome di chi deve svolgerla.\\ In minor misura sono state usate le GitHub Projects boards. Esse aiutano a organizzare e prioritarizzare il lavoro. Si tratta di schede che contengono issues e note che vengono categorizzate in cards su colonne, ogni colonna indica lo stadio di avanzamento del compito.\\
   
    \paragraph{Telegram: }
    E' stato creato un gruppo Telegram per aggiornarsi giornalmente sullo sviluppo del progetto, esprimere eventuali dubbi e avere ulteriori chiarimenti. 
    
    \paragraph{Incontri: }
    Si è deciso di sentirsi una volta a settimana, per risolvere insieme alcuni task rimasti,aiutarsi nelle problematiche riscontrate durante lo sviluppo, fare il chiarimento della situazione. Eventualmente si stilava anche lo Sprint per la settimana successiva, se quello precedente era stato completato. Sono state alternate sedute sia in presenza che online, dove ci si è avvalsi della piattaforma Microsoft Teams.
    
     

\section{Continuous Integration e Automatizzazione}
\label{chap:CI}
In questo capitolo si analizza la parte di integrazione e automatizzazione sia del progetto software che della relazione.\\  Si è pensato di organizzare l'automazione dei compiti più ripetitivi attraverso l'approccio CI/CD, sia per il progetto che per la stesura della relazione.
    \subsection{Relazione di Progetto}
        La relazione è stata scritta su Overleaf, uno strumento di scrittura e pubblicazione collaborativo online.
        Inoltre il progetto Overleaf è stato sincronizzato su un apposito repository in Github, dove poter inserire le modifiche ogni volta. Il branch main ospita la release finale della relazione. 
        
    \subsection{Progetto}
   
        Il branch master è il centro del nostro progetto, perché rispecchia la versione pubblica attualmente in produzione, la versione cioè accessibile ai nostri utenti. L’approccio GitFlow, che abbiamo seguito fedelmente, infatti utilizza due rami principali per gestire il controllo di versione del progetto. Il primo, master che conserva, come detto, tutte le release. Il secondo è il ramo dev, che è fondamentale per lo sviluppo delle prossime versioni e serve come base per le future integrazioni. Master e Dev rappresentano l'ossatura del repository. Essi sono il punto di partenza e di arrivo di tutti gli altri branch.\\
    Per lo sviluppo di ogni issue veniva creato un branch apposito. Al completamento il codice veniva compilato e testato, infine veniva fatta una pull request, dove veniva richiesta la review da parte di tutti gli altri componenti del team per poter includere le proprie modifiche al progetto. Tali modifiche sarebbero poi state inserite sul branch di sviluppo dev. \\
    Periodicamente quindi veniva fatta una release sul branch main. Il codice sul branch dev veniva ritestato e anche i membri del team eseguivano gli opportuni User Acceptance test per verificare le varie funzionalità implementate. 
    Il deploy finale del progetto per la produzione è stato fatto attraverso le Github Pages.
   
        
        





