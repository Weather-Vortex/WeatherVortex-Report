% -*- root: ../main.tex -*-
\chapter{Processo di Sviluppo}


\section{Metodologia di Sviluppo}
Per sviluppare il progetto è stata utilizzato il framework Scrum. 
    \subsection{Scrum}
    Scrum è un framework agile, incrementale e iterativo per lo sviluppo di prodotti, applicazioni e servizi. E' una modalità strutturata e pianificata.
    
    Scrum si basa sull’empirismo, ovvero sul concetto che la conoscenza derivi dall’esperienza e che le decisioni vadano prese alla luce di ciò che si conosce. I tre pilastri che sostengono l’empirismo sono: trasparenza, ispezione e adattamento.
    
    Ovvero, tutti gli aspetti del lavoro devono essere visibili ai responsabili del risultato finale (trasparenza). Per rendere trasparenti questi elementi, il Team Scrum ispeziona di frequente il prodotto mentre lo sta sviluppando (ispezione). Così il processo e il prodotto possono essere adattati immediatamente nel caso di nuove esigenze o di condizioni mutate del mercato (adattamento). 
    
    Perchè questo è il senso di Scrum, far lavorare tutto il team insieme, in modo coordinato e organizzato.
    
    Come tutte le metodologie Agile, si basa sulla divisione del progetto in più fasi, chiamate Sprint.
Ad ogni Sprint il gruppo di lavoro presenta nuove funzionalità, operative e implementabili. Si configura così un sistema iterativo che consente di incrementare poco alla volta, ma molto di frequente, le funzionalità del progetto, verificando costantemente allo stesso tempo l'andamento complessivo.
    

\section{Gestione di Progetto}
In questa sezione viene dettagliatamente spiegato come il progetto è stato organizzato. Gli strumenti e tecnologie con cui è scelto di procedere verranno elencati assieme alla descrizione della metodologia.
    \paragraph{Gant Chart} 
    Da inizio progetto abbiamo stilato un diagramma di Gant, uno strumento utile per la pianificazione dei progetti. Attraverso una panoramica dei compiti programmati, ci ha permesso di venire a conoscenza dei compiti e delle rispettive scadenze.
    
    \paragraph{Licensing} P
    
    \paragraph{Versioning}
    
    \paragraph{GitHub Projects Management}
    descrizione GitHub Projects boards 
    
    \paragraph{Telegram}
    Abbiamo creato un gruppo telegram per aggiornarci giornalmente sullo sviluppo del progetto, esprimere eventuali dubbi e avere ulteriori chiarimenti. 
    
    \paragraph{Microsoft Teams}
    Abbiamo deciso di sentirci una volta a settimana, per risolvere insieme alcuni task rimasti, fare il chiarimento della situazione e stilare lo Sprint per la settimana successiva. Abbiamo alternato sia sedute in presenza che online, dove ci siamo avvalsi della piattaforma Microsoft Teams.
    
     

\section{Continuous Integration e Automatizzazione}
\label{chap:CI}
In questo capitolo si analizza la parte di integrazione e automatizzazione sia del progetto software che della relazione. 
    \subsection{Relazione di Progetto}
        descrizione CI della relazione da Overleaf a GitHub action e la release

    \subsection{Progetto}
        CI implementata per il testing 
        
        





