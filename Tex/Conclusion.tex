% -*- root: ../main.tex -*-

\chapter{Conclusioni}
    
    \section{Commenti Finali}
        
        \subsection{Daniele Tentoni}
        
        Questo progetto mi ha sin da subito motivato moltissimo nel lavorare data la sua natura intrinseca e l'oggetto del sistema da sviluppare. Essendo scout, ho spesso provato un certo astio nei confronti dei differenti providers di previsioni che non si riescono mai a mettere d'accordo e, più ne vengono consultati, più le idee diventano confuse. Con questo progetto ho sperato di creare qualcosa che io possa usare anche in futuro e che possa essere conosciuto anche da altre persone.
        
        L'inizio dello sviluppo, in estate, è andato bene e liscio per poi rallentare drasticamente nel momento in cui sono rincominciate le lezioni. Questo ha fatto calare la mia motivazione. L'uso di tecnologie che io già conoscevo per via del mio lavoro extra universitario e di progetti personali non ha certo aiutato a proseguire a passo spedito.
        
        Penso tuttavia che questo progetto mi abbia aiutato ad avere una migliore visione di cosa voglia dire sviluppare dei sistemi applicativi web al giorno d'oggi, avendo utilizzato tecnologie che vengono tutti i giorni utilizzate da altri milioni di persone al mondo e che sono in costatnte sviluppo.
        
        Il risultato finale del progetto secondo me è buono ma non ottimo. Avrei voluto avere più tempo per poter portare a termine tutte le funzionalità che avevamo avuto in mente, ma con il susseguirsi dello sviluppo, abbiamo notato che avevamo sovrastimato il carico complessivo.
        
        \subsection{Silvia Zandoli}
        
        Questo progetto mi ha dato la possibilità di approfondire le mie conoscenze riguardanti l’utilizzo di NodeJs, Express e MongoDb. Tutte le tecnologie che sono state usate erano per me completamente nuove, ma sono soddisfatta del mio apprendimento. Ho anche compreso come con Vue si riescono a realizzare velocemente e in modo non troppo complesso degli applicativi molto interessanti ed user-friendly. 
        
        Un aspetto importante, non tanto riguardante l’ambito dello Sviluppo Web, che ho potuto imparare è l’importanza dell’organizzazione: collaborare in questo team mi ha incentivato ad organizzare il lavoro, e questo mi ha insegnato un modo migliore di lavorare. A inizio progetto eravamo partiti con in mente qualche funzionalità in più. Poi per una questione di tempi, un membro ha avuto qualche difficoltà in più, non sono stati realizzati.
                Sono comunque soddisfatta del progetto svolto e lo ritengo un'idea innovativa, magari da approfondire in futuro. 

        \subsection{Igor Lirussi}
        
        Al termine del percorso di sviluppo del progetto ritengo di aver acquisito una discreta moltitudine di nuove conoscenze. Sono state impiegate e integrate diverse tecnologie e metodologie anche complesse con cui sono state affinate parecchio le mie competenze nell'ambito. Sicuramente c'è ancora spazio per approfondire molti argomenti, ma nel complesso reputo sia servito a formare una base di conoscenza a tutto tondo. Il lavoro di gruppo ha permesso, infatti, di scambiarci nozioni a vicenda e ne sono più che contento, seppur mi rincresce non essere stato presente a volte come avrei voluto. Molte tecnologie usate sono state per me una scoperte \emph{in-itinere}, in quanto completamente nuove. Ritengo comunque il progetto sia stato particolarmente ambizioso e il risultato ottenuto a mio parere è più che soddisfacente. Infine, spero questo lavoro possa rappresentare una buona base di partenza per un eventuale sviluppo futuro. 
    
    \section{Sviluppi Futuri}
    
    In futuro, il progetto potrà essere migliorato nelle sue parti lasciando spazio a nuove tecnologie e, auspicabilmente, anche ad una commercializzazione. 
    Le aggiunte che abbiamo individuato sono sia di natura grafica che funzionale, ma, a causa delle tempistiche ridotte e del carico di lavoro, sono state demandate ad un'implementazione futura.
    
    Eventualmente la parte funzionale si può estendere creando nuove pagine per l'utente finale, con sezioni quali "notizie", "video degli utenti", "allerte meteo". Inoltre nuove funzioni possono sempre essere aggiunte al backend per migliorare la parte di aggregazione delle varie previsioni con machine learning, ponderamento in base ai feedback o allo storico delle condizioni meteo reali. Può essere implementata una parte amministrativa del sito e integrata in maniera non invasiva della pubblicità. 
    
    La parte grafica invece può essere ulteriormente migliorata con animazioni e elementi accattivanti, eventualmente che rispecchino le attuali condizioni meteo. Inoltre mappe interattive possono aiutare l'utente ad avere un'idea più chiara della situazione geografica. Eventualmente anche temi personalizzati possono essere sviluppati per rispecchiare i gusti degli utilizzatori.
    
    Negli sviluppi futuri includiamo anche un interfacciamento con i social maggiormente diffusi, oramai essenziali allo sviluppo di un business. In questo modo si potrebbe automatizzare la creazione di post, facilitare l'interazione tra gli utenti, lo sviluppo di una community e l'esposizione della piattaforma.
    
    Inoltre, come possibili spin-off sono state individuate delle aree tematiche spesso ignorate in cui le previsioni sono necessarie, ad esempio previsioni per lo sport, quali venti per i praticanti di parapendio o correnti marine per i surfisti. Anche qui è altamente necessario l'utilizzo di device IoT.
    
    A riguardo, in futuro si potrebbe pensare ad una commercializzazione di un device integrato "ready to use" per delle previsioni più personali.
    
    In conclusione in futuro il progetto offre numerose opportunità di ampliamento, con interessanti prospettive sia dal lato tecnologico che dal lato business.  
    
