% -*- root: ../main.tex -*-

% Esporre lo stato di funzionamento effettivo del sistema progettato ad elaborato concluso. Per ciascuna delle funzionalità salienti devono essere tabellate e discusse le performance riscontrate mediante opportuni test eseguiti in fase di validazione del progetto. I tempi di esecuzione/comunicazione devono essere accompagnati dalle caratteristiche dell'hardware sul quale è eseguito il software. Qualora l'elaborato includa algoritmi innovativi, indicarne la complessità computazionale (avendo cura di esporre lo pseudo codice nella sezione Implementazione).

\chapter{Testing e Performance}

Lo sviluppo delle applicazioni web a livello professionale richiede la qualità e la robustezza del codice che è stato prodotto. Per avere delle metriche concrete su tali metriche, è necessario eseguire intensivamente ed estensivamente test sul prodotto, sia automatici che manuali.

\section{Unit Test}

Le parti più importanti inerenti la logica di business sono stati testati con gli unit test per migliorarne la robustezza e l'affidabilità. Molte componenti dell'applicativo Server sono state sviluppate con un approccio Test Driven Development e sono state controllate molte volte prima di essere messe in produzione.

\par In generale si è cercato di tenere un approccio sistematico per affrontare i problemi:

\begin{itemize}

    \item Si cercava di modellare in modo il più reale possibile le entità che si sarebbero manipolate successivamente, cercando di capire il dominio del problema e quali fossero tutti i dati di cui si avesse bisogno;
    
    \item Successivamente si sviluppava il componente che sarebbe stato delegato a interagire principalmente con la base dati delle entità appena modellate. Creati i test automatici per controllare che i dati venissero manipolati correttamente a seconda delle necessità, solamente in un secondo momento, dopo aver appurato il fallimento dei test appena scritti, si sarebbe proceduto a scrivere il codice necessario per farli riuscire;
    
    \item Allo stesso modo, si sviluppavano anche i componenti della logica di business.
    
    \item Infine, al manifestarsi di ulteriori problemi e malfunzionamenti, si è sempre cercato innanzitutto di comprendere se ci fossero stati dei problemi nell'implementazione dei test e che le loro funzionalità fossero quelle che ci si aspettava. In tal caso, allora si procedeva alla stesura di altri test automatici che permettessero di riprodurre il problema e risolverlo.
    
\end{itemize}

 \subsection{Backend}
 
 Il backend utilizza il testing stack Mocha + Chai + Sinon. 
Mocha è il testing framework ricco e semplice che consente di creare facilmente degli unit test eloquenti. Chai è la libreria di assertions BDD/TDD che consente di creare le asserzioni multi-paradigma per i test. Tali asserzioni rendono i test più leggibili e forniscono messaggi di errore migliori. 
Sinon è la libreria per creare e gestire i test spies, stubs e mocks da utilizzare nei test. Nel nostro caso è stato utilizzato soltanto per simulare l'invio delle email.

Durante lo sviluppo del backend è stata adoperata una collection Postman per testare tutte le varie chiamate al server anche in assenza dell'applicativo client definitivo, in attesa di sviluppo. 

\section{User Assurance Test}

Alla fine della realizzazione del progetto tutti i membri del team hanno testato il sistema integrale per decidere se poteva essere accettato o meno. Ciò ha permesso di riconoscere alcune inconsistenze e aspetti poco chiari e di sistemarli. Inoltre la piattaforma è stata fatta provare a degli utenti che non hanno seguito lo sviluppo (e che quindi non sono soggetti ad un bias cognitivo come quello sviluppato da noi programmatori).