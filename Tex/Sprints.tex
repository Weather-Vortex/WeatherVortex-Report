% -*- root: ../main.tex -*-
\chapter{Sprints}

\section{Svolgimento}
Gli sprint sono stati portati avanti nel seguente modo:
    \paragraph{Sprint Planning}
        Pianificazione a inizio sprint degli obiettivi, tempistiche e responsabilità nel periodo dello sprint corrente. Diviso in due parti:
        \begin{itemize}
        \item\textbf{parte 1} 
            Viene raffinato e rivisto il product backlog, viene effettuata la scelta dello sprint goal (what).
        \item\textbf{parte 2}
            Si decidono gli item e viene raffinato come implementarli (how). Effettuato con solo il team senza la figura del product owner
        \end{itemize}
    \paragraph{[Iterativo] Daily scrum} Breve meeting svolto giornalmente. Viene utilizzato per gli aggiornamenti sull'andamento del progetto, senza scendere nei dettagli implementativi.
    \paragraph{[Occasionale] Pair Programming } Utilizzato per risolvere problemi che causano il blocco di un componente del team per parecchio tempo su una issue.
    \paragraph{Meeting finale}
        Riflessioni e considerazioni finali sullo spint passato. Suggerimenti per migliorare il prossimo. Diviso in tre parti: 
        \begin{itemize}
        \item\textbf{Product backlog refinement} aggiunta di dettagli e riordino del product backlog
        \item\textbf{Sprint review} è stato ispezionato l'incremento, il Minimum Viable Product o di risultati sul processo. Discernere cosa è stato fatto e cosa no
        \item\textbf{Retrospettiva} Considerazioni sul team stesso e sui miglioramenti per il prossimo sprint. 
        \end{itemize}
        
        

\section{Sprint 0}
All'interno dello sprint 0 il focus è stato ...
\paragraph{Deliverables} 
\begin{itemize}
    \item architettura base server
    \item scheletro relazione
\end{itemize}


