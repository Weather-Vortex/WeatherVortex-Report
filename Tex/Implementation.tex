% -*- root: ../main.tex -*-

% Esporre i principali problemi affrontati durante l'effettiva realizzazione delle componenti hardware/software e illustrare le soluzioni implementative adottate. Se l'elaborato ha previsto l'utilizzo di tecnologie già disponibili sul mercato, discuterne brevemente le caratteristiche e motivarne l'adozione rispetto ad altre soluzioni assimilabili. NOTA: in questa sezione devono essere riportate esclusivamente le porzioni di codice ritenute particolarmente significative.

\chapter{Implementazione}
\section{Server}
Implementazione del server\\
\textit{Commento: questo secondo me è da fare insieme, ognuno può aggiungere qualcosa della propria parte. Magari scrive uno e l'altro se vuol dire qualcosa anche della sua parte la aggiunge.}

\subsection{Autenticazione}
\textit{TODO: Da finire, lo fa Silvia questa subsection, mettere anche un pezzo di codice}.\\ 
Per l'autenticazione è stato deciso di utilizzare il sistema a token JWT, firmati dal backend. Al momento
del login il server rilascia un token al client contenente dei dati
utili all’autenticazione; in questa maniera, durante le successive richieste,
il client invierà anche il token ricevuto, e il server lo controllerà per verificare
se è valido e quindi autorizzare il client. In questa maniera il server
diventa stateless, non avendo bisogno di salvare le informazioni relative
alle sessioni dei vari client collegati.\\

Si è avuta la necessità anche di gestire alcuni dati sensibili dell'utente, come le password. E' stata utilizzata bcrypt, una funzione di hashing per salvare le password in maniera sicura
nel database; viene utilizzato al momento della registrazione di un utente
quando si deve salvare la password nel db (viene salvato il suo hash con il
relativo valore di sale), e al login quando bisogna confrontare la password
ricevuta dall’utente con quella salvata. 



\section{Client}
Implementazione del client

TODO parlare dei vari componenti che compongono ogni pagina view
parlare del vue router

\section{Centralina}
Implementazione centralina


\section{Prodotto Finale}
TODO: Screen di esempio dell'applicazione