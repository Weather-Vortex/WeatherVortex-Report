% -*- root: ../main.tex -*-

% In questa sezione esporre brevemente i requisiti a cui il sistema proposto deve rispondere, concentrando l'attenzione sugli aspetti più rilevanti e facendo eventualmente uso di opportuni diagrammi di alto livello.

\chapter{Analisi dei Requisiti}
In questa fase sono stati individuati i \textbf{requisiti del sistema}, partendo dalle descrizioni di alto livello, ottenute dal committente durante il \textbf{knowledge crunching e i focus group}. Successivamente si è proceduto con un raffinamento che ha portato alla definizione di requisiti più \textbf{specifici}, \textbf{chiari} e \textbf{strutturati}.
	
	\section{Requisiti di Business}
	Si definiscono di seguito le aspettative del cliente e i requisiti che il prodotto dovrà soddisfare, espressi con una terminologia ad elevato livello astrattivo.
        \begin{itemize}
        \item Il prodotto dovrà costituire un \textbf{unico servizio} per consultare e paragonare diverse previsioni meteo provenienti da vari provider internazionali.
        \item L'applicativo dovrà permettere di visualizzare all'utente le \textbf{previsioni aggregate}. 
            
            \item Il prodotto dovrà facilitare le decisioni dell'utente e permettere di arrivare facilmente ad una \textbf{previsione affidabile}. 
            \item Opzionalmente il prodotto dovrà fornire un sistema di notifiche giornaliere. 
            
        \end{itemize}
	
	\section{Requisiti Utente}
	Di seguito vengono riportate le richieste mosse dal cliente in maniera informale evitando termini tecnici, successivamente tali richieste saranno formalizzate per quanto possibile.
	Il prodotto dovrà fornire: 
		\begin{itemize}
            \item L'accesso al sistema da \textbf{qualunque dispositivo} munito di connessione alla rete, anche mobile o tablet.
            \item Tutte le operazioni necessarie per consultare previsioni meteo.
            \item \textbf{Prestazioni adeguate} per il caricamento.
            \item Un' interfaccia intuitiva e usabile.
            \item Possibilità di creare un account in cui gestire le proprie informazioni.
            \item Possibilità di registrare una centralina.
            \item Possibilità di valutare l'affidabilità dei provider.
            \item Un'\textbf{esperienza personalizzata} e user-friendly.
        \end{itemize}
        \subsection{User stories}
        Di seguito sono riportate tutte le \textbf{user stories} formulate assieme agli utilizzatori
        \begin{itemize}
            \item Come \textbf{utente}
            voglio:
            \begin{itemize}
                \item poter \textbf{visualizzare le previsioni odierne} per una località specifica o per la propria posizione attuale.
                \item poter \textbf{visualizzare le previsioni relative ai prossimi giorni} per una località specifica o per la propria posizione attuale.
                \item poter \textbf{impostare la fiducia per ogni servizio} usato e fornire un feedback, con la possibilità anche di eliminarlo.
                \item \textbf{specificare la località preferita} di cui vedere le previsioni non appena viene visualizzata l'area riservata, oppure modificare o rimuovere quest'ultima.
                \item \textbf{aggiungere un device}, registrandolo come centralina meteo a cui richiedere le previsioni con la possibilità di modificarlo o rimuoverlo.
                \item \textbf{ricevere notifiche ad una certa ora} con le previsioni giornaliere sulla località preferita.
                \item \textbf{visualizzare le statistiche} e indici di affidabilità di un provider in base ai feedback degli altri utenti.
                
            \end{itemize}
        \end{itemize}
            
	    
	\section{Requisiti Funzionali} %obbligatori, desiderabili e opzionali
	Di seguito si riportano i requisiti funzionali, descrivendo in modo più dettagliato le funzionalità che deve avere l'applicazione:
	 \begin{itemize}
	 \item L'applicativo dovrà fornire la possibilità di aggiungere una previsione inserendo una località specifica o tramite coordinate geografiche.
	\item L'applicativo dovrà \textbf{ridurre il tempo necessario} per consultare più previsioni.
	\item L'applicazione dovrà fornire i seguenti dettagli delle previsioni: condizione meteo, temperatura, pressione, velocità del vento e direzione, quantità di pioggia e umidità.
	\item Il sistema dovrà produrre le previsioni aggregate come media di tutte le previsioni dei vari provider internazionali.
	\item Il sistema deve offrire previsioni correnti e fino a tre giorni.
	\item L'applicativo dovrà fornire per ogni provider meteo un indice di affidabilità, dato dalla media dei ratings dei suoi feedbacks.
	\item L'applicativo dovrà fornire maggiori funzionalità per un utente registrato quali poter aggiungere, aggiornare oppure rimuovere le centraline,la località preferità e eventuali feedbacks sui provider.
	\item Il sistema fornirà un sistema di notifiche che invii ad una certa ora le previsioni giornaliere sulla località preferita dell'utente via email.
	\item L'applicativo dovrà salvarsi in memoria le ultime previsioni in modo da risparmiare tempo su richieste successive della stessa zona. 
	\item L'applicativo dovrà mostrare degli alert in caso di errori (ad esempio: quando le previsioni non sono state caricate correttamente, c'è stato un errore durante la registrazione o il login, oppure durante l'aggiunta, o la cancellazione di una centralina o di un feedback, etc).
	\item L'applicativo dovrà mostrare anche degli alert qualora le operazioni siano avvenute con successo per facilitare l'esperienza utente (Ad esempio: la registrazione o il login sono avvenuti con successo, la centralina o il feedback sono stati aggiunti correttamente etc.).
	\item Il prodotto dovrà fornire un'interfaccia responsive e usabile.
	\end{itemize}
        \subsection{Applicativo web}
        L'applicativo web deve consentire ad ogni utente di accedere con una serie di credenziali, e dividere le operazioni consentite in base ai permessi concessi.
        Il sito è suddiviso pagine :
        \begin{itemize}
            \item \textbf{Pagine di autenticazione}
                Pagina di login per effettuare l'accesso e pagina di registrazione per i nuovi utenti.
            \item \textbf{home}
                La pagina principale in cui è possibile registrarsi, accedere, visualizzare le previsioni. E' visibile da chiunque.
            \item \textbf{Pagina Profilo}: E' il profilo privato dell'utente a cui si può accedere solo se si è registrati. Permette di visualizzare e eventualmente modificare le proprie informazioni, visualizzare o cancellare i propri feedbacks, visualizzare, modificare o cancellare le proprie centraline.
            \item \textbf{Pagina Profilo Pubblico}: Serve per visualizzare le informazioni di un altro utente. E' visibile da chiunque.
             \item \textbf{Pagina Forecast}: Pagina dove è possibile visualizzare il meteo corrente e le previsioni dei prossimi tre giorni, sia aggregate che non. E' visibile da chiunque.
              \item \textbf{Pagina Feedbacks}: Pagina dove è possibile visualizzare i feedbacks dei vari provider e consultare i loro indici di affidabilità. E' visibile da chiunque, ma si possono aggiungere feedbacks solo se registrati. 
              \item \textbf{Pagina About}: Pagina dove è possibile visualizzare informazioni sugli sviluppatori del progetto e sullo scopo dell'applicativo, con un modulo di contatto agli admin.
              
            
        \end{itemize}
        
    
        
	\section{Requisiti non Funzionali}	
    Il sistema dovrà rispettare i seguenti requisiti non funzionali per un'alta \textbf{qualità}:
 
            \begin{itemize}
                \item \textbf{Reattività}: L'applicazione dovrà essere reattiva e fornire un basso tempo di risposta, tutte le azioni dovranno essere processate istantaneamente e le informazioni dovranno essere aggiornate subito.
                \item \textbf{Modularità}:L'applicazione dovrà essere modulabile, con lo scopo di aggiungere senza problemi eventuali nuove funzionalità.
                \item \textbf{Scalabilità}: L'applicativo dovrà necessariamente consentire di aumentare o diminuire il numero di utenti gestiti senza particolari problemi. 
                \item \textbf{Disponibilità}:L'applicazione dovrà essere disponibile 24/7. L'utilizzo di più api per il meteo , oltre allo scopo di fornire l'aggregazione delle previsioni, ha anche quello di mantenere le previsioni disponibili anche se qualche provider è momentaneamente down.
                \item L'applicazione dovrà lavorare senza errori, essere veloce e fornire una buona "user experience".
            \end{itemize}
        

	\section{Requisiti Implementativi}
	Si definiscono i vincoli a livello tecnologico e di implementazione che sarà necessario rispettare nella realizzazione dell'applicativo.
	
	\subsection{Metodologie e Tecnologie Utilizzate}
	In quanto progetto sviluppato per l'esame di Applicazioni e Servizi Web è necessario che le tecnologie usate rispettino gli standard di insegnamento. 
	