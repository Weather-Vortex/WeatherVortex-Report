% -*- root: ../main.tex -*-

% In questa sezione esporre brevemente i requisiti a cui il sistema proposto deve rispondere, concentrando l'attenzione sugli aspetti più rilevanti e facendo eventualmente uso di opportuni diagrammi di alto livello.

\chapter{Analisi dei Requisiti}
In questa fase sono stati individuati i \textbf{requisiti del sistema}, partendo dalle descrizioni di alto livello, ottenute dal committente durante il \textbf{knowledge crunching e i focus group}. Successivamente si è proceduto con un raffinamento che ha portato alla definizione di requisiti più \textbf{specifici}, \textbf{chiari} e \textbf{strutturati}.
	
	\section{Requisiti di Business}
	Si definiscono di seguito le aspettative del cliente e i requisiti che il prodotto dovrà soddisfare, espressi con una terminologia ad elevato livello astrattivo.
        \begin{itemize}
            \item L'applicativo dovrà \textbf{ridurre il tempo necessario} per consultare più previsioni.
            \item Il prodotto dovrà ...
            \item Opzionalmente il prodotto dovrà \textbf{...} 
        \end{itemize}
	
	\section{Requisiti Utente}
	Di seguito vengono riportate le richieste mosse dal cliente in maniera informale evitando termini tecnici, successivamente tali richieste saranno formalizzate per quanto possibile.
	Il prodotto dovrà fornire: 
		\begin{itemize}
            \item l'accesso al sistema da \textbf{qualunque dispositivo} munito di connessione alla rete, anche mobile o tablet
            \item \textbf{prestazioni adeguate} per il caricamento
            \item un interfaccia intuitiva
            \item account
            \item possibilità di registrare una centralina
        \end{itemize}
        \subsection{User stories}
        Di seguito sono riportate tutte le \textbf{user stories} formulate assieme agli utilizzatori
        \begin{itemize}
            \item Come \textbf{utente}
            voglio:
            \begin{itemize}
                \item poter \textbf{visualizzare le previsioni odierne} per una località specifica o per la propria posizione attuale
                \item poter \textbf{visualizzare le previsioni relative ai prossimi giorni} per una località specifica o per la propria posizione attuale
                \item poter \textbf{impostare la fiducia per ogni servizio} usato e fornire un feedback
                \item \textbf{specificare la località preferita} di cui vedere le previsioni non appena viene visualizzata l'area riservata.
                \item \textbf{aggiungere un device}, registrandolo come centralina meteo a a cui richiedere le previsioni
                \item \textbf{ricevere notifiche ad una certa ora} con le previsioni giornaliere sulla località preferita
                \item \textbf{visualizzare le statistiche} e grafici di affidabilità di un provider in base ai feedback degli utenti.
                
            \end{itemize}
        \end{itemize}
            
	    
	\section{Requisiti Funzionali} %obbligatori, desiderabili e opzionali
	Di seguito si riportano i requisiti non funzionali:
	    \subsection{Casi d'uso}
        	 Diagramma dei casi d'uso
                    
        \subsection{Applicativo web}
        L'applicativo web deve consentire ad ogni utente di accedere con una serie di credenziali, e dividere le operazioni consentite in base ai permessi concessi.
        Il sito è suddiviso pagine :
        \begin{itemize}
            \item \textbf{pagina di login}
                La pagina utilizzata per effettuare l'accesso
            \item \textbf{home}
                La pagina principale in cui è possibile:
                \begin{itemize}
                    \item accedere
                    \item registrarsi
                    \item consultare le previsioni
                \end{itemize}
            
        \end{itemize}
        
    
        
	\section{Requisiti non Funzionali}	
    Il sistema dovrà rispettare i seguenti requisiti non funzionali per un'alta \textbf{qualità}:
        \subsection{Di Sistema}
            \begin{itemize}
                \item \textbf{Reattività}: 
                    \begin{itemize}
                        \item l'utente non deve percepire \textbf{ritardi} 
                        \item le notifiche devono ..
                    \end{itemize}
                \item \textbf{Scalabilità}: L'applicativo deve necessariamente consentire di aumentare o diminuire il numero di utenti gestiti senza particolari problemi. 
            \end{itemize}
        

	\section{Requisiti Implementativi}
	Si definiscono i vincoli a livello tecnologico e di implementazione che sarà necessario rispettare nella realizzazione dell'applicativo.
	
	\subsection{Metodologie e Tecnologie Utilizzate}
	In quanto progetto sviluppato per l'esame di Applicazioni e Servizi Web è necessario che le tecnologie usate rispettino gli standard di insegnamento. 
	