% -*- root: ../main.tex -*-

% Riassumere le soluzioni presenti in letteratura inerenti al problema in esame. Per ciascuna, discutere le principali diversità o affinità rispetto al progetto presentato. Nel caso non siano presenti soluzioni direttamente comparabili a quella presentata descrivere comunque le principali tecniche note per affrontare la tematica trattata. Le soluzioni esposte devono essere corredate degli opportuni riferimenti bibliografici. Nel caso si tratti di soluzioni già operative sul mercato, devono essere indicate le fonti (online) dove poter accedere al servizio o approfondirne i contenuti.

\chapter{Stato dell'Arte}
Il topic scelto in letteratura ha ricevuto molte attenzioni per via della stessa natura delle previsioni meteo, fortemente legate al machine learning. Tuttavia per quanto riguarda l'aggregazione di esse non risultano presenti particolari ricerche.
\par Dalle informazioni che abbiamo raccolto su studi e applicazione di \textbf{idee similari} alla nostra è emerso quanto segue:

\begin{itemize}

    \item \textbf{Aggregazione}: sono presenti alcuni siti similari ma offrono funzionalità molto limitate, è possibile vedere solo le previsioni odierne o è impossibile avere un account che permetta di scegliere quanta fiducia dare a diversi servizi.
    
    \begin{itemize}
    
        \item Il servizio https://www.metaweather.com/it/ fornisce un servizio di aggregazione di dati simile a quello che vogliamo ottenere con il nostro progetto, ma la visualizzazione delle recensioni dei servizi meteo non è pubblica e non viene data la possibilità di aggiungere centraline meteo.
        
        \item Google sta progettando con successo una Deep Neural Network per predire più efficacemente le condizioni meteo combinando esclusivamente i dati delle previsioni meteo passate negli ultimi anni. Applicando poi questi algoritmi alle previsioni delle ultime ore o giorni, senza alcuna conoscenza dei modelli fisici adottati fino adesso per la previsione delle condizioni meteo, è in grado di determinare talvolta con estrema accuratezza le condizioni future fino a 8 ore: https://neurohive.io/en/news/google-s-deep-neural-network-makes-detailed-weather-forecasts/
        
    \end{itemize}
    
    \item \textbf{Personalizzazione}: non sono presenti siti, se non prototipali, che permettano di registrare una centralina personale da aggiungere alle fonti per le previsioni. In genere gli appassionati adottano delle soluzioni pensate appositamente per le proprie centraline che non comunichino con altre centraline di altri appassionati oppure con servizi meteo esistenti. Talvolta potrebbero esserci progetti che raccolgano dati da centraline multiple, ma ciò è mirato a produrre un unica previsione meteo incrociando i dati dalle fonti facendo perdere il concetto di centralina e previsione per la singola entità, in favore della previsione meteo per una determinata zona come sistema di centraline.
    
\end{itemize}
 

