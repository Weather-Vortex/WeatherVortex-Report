% -*- root: ../main.tex -*-

% Riassumere le soluzioni presenti in letteratura inerenti al problema in esame. Per ciascuna, discutere le principali diversità o affinità rispetto al progetto presentato. Nel caso non siano presenti soluzioni direttamente comparabili a quella presentata descrivere comunque le principali tecniche note per affrontare la tematica trattata. Le soluzioni esposte devono essere corredate degli opportuni riferimenti bibliografici. Nel caso si tratti di soluzioni già operative sul mercato, devono essere indicate le fonti (online) dove poter accedere al servizio o approfondirne i contenuti.

\chapter{Stato dell'Arte}
Il topic scelto in letteratura ha ricevuto molte attenzioni per via della stessa natura delle previsioni meteo, fortemente legate al machine learning. Tuttavia per quanto riguarda l'aggregazione di esse non risultano presenti particolari ricerche.
Dalle informazioni che abbiamo raccolto su studi e applicazione di \textbf{idee similari} alla nostra è emerso quanto segue:

\begin{itemize}
    \item \textbf{Aggregazione}: sono presenti alcuni siti similari ma offrono funzionalità molto limitate, è possibile vedere solo le previsioni odierne o è impossibile avere un account che permetta di scegliere quanta fiducia dare a diversi servizi. 
    \item \textbf{Personalizzazione}: non sono presenti siti, se non prototipali, che permettano di registrare una centralina personale da aggiungere alle fonti per le previsioni.
\end{itemize}
 

